\documentclass[a4paper,12pt]{article}

% Pacotes Necessários
\usepackage[utf8]{inputenc}
\usepackage[brazilian]{babel}
\usepackage{graphicx}
\usepackage{hyperref}
\usepackage{geometry}
\geometry{a4paper, margin=2.5cm}

\title{Proposta de Desenvolvimento do Sistema Saba Arte}
\author{Wanderlei Rodrigo Pereira}
\date{\today}

\begin{document}

\maketitle

\section*{Introdução}
Esta proposta apresenta o desenvolvimento de um site completo para um artista, com as seguintes funcionalidades principais: 
\begin{itemize}
    \item Galeria de pinturas com imagens de alta qualidade e descrições detalhadas.
    \item Opção para compra dos quadros diretamente pelo site.
    \item Resumo da trajetória do artista, destacando sua formação e estilo artístico.
    \item Design moderno, responsivo e otimizado para SEO.
\end{itemize}

\section*{Estrutura do Site}
O site será composto pelas seguintes páginas principais:
\begin{enumerate}
    \item \textbf{Página Inicial:}
    \begin{itemize}
        \item Imagem de destaque com uma das principais obras ou o artista em ação.
        \item Breve introdução sobre o artista e seu trabalho.
    \end{itemize}
    \item \textbf{Galeria:}
    \begin{itemize}
        \item Exibição das pinturas com imagens, descrições, preços e possibilidade de filtro por categoria (e.g., paisagens, retratos).
    \end{itemize}
    \item \textbf{Comprar:}
    \begin{itemize}
        \item Integração com sistemas de pagamento como PayPal ou Stripe.
        \item Carrinho de compras para gerenciar os pedidos.
    \end{itemize}
    \item \textbf{Sobre o Artista:}
    \begin{itemize}
        \item Resumo detalhado da trajetória do artista, incluindo fotos do ateliê.
        \item Destaque para prêmios, exposições e estilo artístico.
    \end{itemize}
    \item \textbf{Contato:}
    \begin{itemize}
        \item Formulário de contato simples e funcional.
        \item Links para redes sociais e localização.
    \end{itemize}
\end{enumerate}

\section*{Tecnologias Utilizadas}
Para garantir desempenho e escalabilidade, o site será desenvolvido com as seguintes tecnologias:
\begin{itemize}
    \item \textbf{Frontend:} React.js ou Next.js para um design dinâmico e responsivo.
    \item \textbf{Backend:} Node.js com Express ou Django para gerenciar funcionalidades e comunicação com o banco de dados.
    \item \textbf{Banco de Dados:} PostgreSQL ou MongoDB para armazenamento eficiente das informações de pinturas e pedidos.
    \item \textbf{Estilo:} Tailwind CSS para criar um layout moderno e responsivo.
    \item \textbf{Hospedagem:} Vercel ou Netlify para o frontend; AWS ou Render para o backend.
\end{itemize}

\section*{Funcionalidades Adicionais}
\begin{itemize}
    \item Painel de administração para o artista gerenciar pinturas e pedidos.
    \item Site otimizado para dispositivos móveis e navegadores modernos.
    \item SEO configurado para aumentar a visibilidade em motores de busca.
    \item Sistema de notificações via e-mail para confirmação de compras.
\end{itemize}

\section*{Cronograma de Desenvolvimento}
\begin{itemize}
    \item \textbf{Semana 1-2:} Planejamento e estruturação inicial do projeto.
    \item \textbf{Semana 3-4:} Desenvolvimento do frontend com React.js ou Next.js.
    \item \textbf{Semana 5-6:} Configuração do backend e integração com banco de dados.
    \item \textbf{Semana 7:} Implementação de sistemas de pagamento e carrinho de compras.
    \item \textbf{Semana 8:} Testes, ajustes finais e lançamento.
\end{itemize}

\section*{Orçamento}
O orçamento estimado para este projeto será detalhado conforme a reunião inicial com o artista para alinhar expectativas e escopo final.

\section*{Conclusão}
Esta proposta visa entregar um site completo e funcional que atenda às necessidades do artista, proporcionando uma experiência agradável tanto para o criador quanto para os visitantes.

\vfill
\noindent Contato: \href{mailto:seuemail@exemplo.com}{seuemail@exemplo.com} 

\end{document}
